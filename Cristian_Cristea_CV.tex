\documentclass[10pt, a4paper]{article}

% Packages:
\usepackage[
    ignoreheadfoot, % set margins without considering header and footer
    top=1.25 cm, % seperation between body and page edge from the top
    bottom=1.25 cm, % seperation between body and page edge from the bottom
    left=1.25 cm, % seperation between body and page edge from the left
    right=1.25 cm, % seperation between body and page edge from the right
    footskip=0.625 cm, % seperation between body and footer
    % showframe % for debugging 
]{geometry} % for adjusting page geometry
\usepackage[explicit]{titlesec} % for customizing section titles
\usepackage{tabularx} % for making tables with fixed width columns
\usepackage{array} % tabularx requires this
\usepackage[dvipsnames]{xcolor} % for coloring text
\definecolor{primaryColor}{RGB}{143, 9, 39} % define primary color
\usepackage{enumitem} % for customizing lists
\usepackage{fontawesome5} % for using icons
\usepackage{amsmath} % for math
\usepackage[
    pdftitle={Cristian Cristea's CV},
    pdfauthor={Cristian Cristea},
    pdfcreator={LaTeX with RenderCV},
    colorlinks=true,
    urlcolor=primaryColor
]{hyperref} % for links, metadata and bookmarks
\usepackage[pscoord]{eso-pic} % for floating text on the page
\usepackage{calc} % for calculating lengths
\usepackage{bookmark} % for bookmarks
\usepackage{lastpage} % for getting the total number of pages
\usepackage{changepage} % for one column entries (adjustwidth environment)
\usepackage{paracol} % for two and three column entries
\usepackage{ifthen} % for conditional statements
\usepackage{needspace} % for avoiding page brake right after the section title
\usepackage{iftex} % check if engine is pdflatex, xetex or luatex

% Ensure that generate pdf is machine readable/ATS parsable:
\ifPDFTeX
    \input{glyphtounicode}
    \pdfgentounicode=1
    \usepackage[T1]{fontenc}
    \usepackage[utf8]{inputenc}
    \usepackage{lmodern}
\fi

\usepackage[default, type1]{sourcesanspro} 

% Some settings:
\AtBeginEnvironment{adjustwidth}{\partopsep0pt} % remove space before adjustwidth environment
\pagestyle{empty} % no header or footer
\setcounter{secnumdepth}{0} % no section numbering
\setlength{\parindent}{0pt} % no indentation
\setlength{\topskip}{0pt} % no top skip
\setlength{\columnsep}{0.15cm} % set column seperation
\makeatletter
\let\ps@customFooterStyle\ps@plain % Copy the plain style to customFooterStyle
\patchcmd{\ps@customFooterStyle}{\thepage}{
    \color{gray}\textit{\small Cristian Cristea's Curriculum Vitae (Résumé) -- Page \thepage{} of \pageref*{LastPage}}
}{}{} % replace number by desired string
\makeatother
\pagestyle{customFooterStyle}

\titleformat{\section}{
    % avoid page braking right after the section title
    \needspace{4\baselineskip}
    % make the font size of the section title large and color it with the primary color
    \Large\color{primaryColor}
}{
}{
}{
    % print bold title, give 0.15 cm space and draw a line of 0.8 pt thickness
    % from the end of the title to the end of the body
    \textbf{#1}\hspace{0.15cm}\titlerule[0.8pt]\hspace{-0.1cm}
}[] % section title formatting

\titlespacing{\section}{
    % left space:
    -1pt
}{
    % top space:
    0.3 cm
}{
    % bottom space:
    0.2 cm
} % section title spacing

% \renewcommand\labelitemi{$\vcenter{\hbox{\small$\bullet$}}$} % custom bullet points
\newenvironment{highlights}{
    \begin{itemize}[
        topsep=0.1 cm,
        parsep=0.1 cm,
        partopsep=0pt,
        itemsep=0pt,
        leftmargin=0 cm + 10pt
    ]
}{
    \end{itemize}
} % new environment for highlights

\newenvironment{highlightsforbulletentries}{
    \begin{itemize}[
        topsep=0.1 cm,
        parsep=0.1 cm,
        partopsep=0pt,
        itemsep=0pt,
        leftmargin=10pt
    ]
}{
    \end{itemize}
} % new environment for highlights for bullet entries


\newenvironment{onecolentry}{
    \begin{adjustwidth}{
        0.2 cm + 0.00001 cm
    }{
        0.2 cm + 0.00001 cm
    }
}{
    \end{adjustwidth}
} % new environment for one column entries

\newenvironment{twocolentry}[2][]{
    \onecolentry
    \def\secondColumn{#2}
    \setcolumnwidth{\fill, 3.5 cm}
    \begin{paracol}{2}
}{
    \switchcolumn \raggedleft \secondColumn
    \end{paracol}
    \endonecolentry
} % new environment for two column entries

\newenvironment{threecolentry}[3][]{
    \onecolentry
    \def\thirdColumn{#3}
    \setcolumnwidth{1 cm, \fill, 3.5 cm}
    \begin{paracol}{3}
    {\raggedright #2} \switchcolumn
}{
    \switchcolumn \raggedleft \thirdColumn
    \end{paracol}
    \endonecolentry
} % new environment for three column entries

\newenvironment{header}{
    \setlength{\topsep}{0pt}\par\kern\topsep\centering\color{primaryColor}\linespread{1.5}
}{
    \par\kern\topsep
} % new environment for the header

\newcommand{\placelastupdatedtext}{% \placetextbox{<horizontal pos>}{<vertical pos>}{<stuff>}
  \AddToShipoutPictureFG*{% Add <stuff> to current page foreground
    \put(
        \LenToUnit{\paperwidth-1.25 cm-0.2 cm+0.05cm},
        \LenToUnit{\paperheight-0.625 cm}
    ){\vtop{{\null}\makebox[0pt][c]{
        \small\color{gray}\textit{Last updated in March 2025}\hspace{\widthof{Last updated in March 2025}}
    }}}%
  }%
}%

% save the original href command in a new command:
\let\hrefWithoutArrow\href

% new command for external links:
\renewcommand{\href}[2]{\hrefWithoutArrow{#1}{\ifthenelse{\equal{#2}{}}{ }{#2 }\raisebox{.15ex}{\footnotesize \faExternalLink*}}}


\begin{document}
    \newcommand{\AND}{\unskip
        \cleaders\copy\ANDbox\hskip\wd\ANDbox
        \ignorespaces
    }
    \newsavebox\ANDbox
    \sbox\ANDbox{}

    \placelastupdatedtext
    \begin{header}
        \fontsize{30 pt}{30 pt}
        \textbf{Cristian Cristea}

        \vspace{0.5 cm}

        \normalsize
        \mbox{{\footnotesize\faMapMarker*}\hspace*{0.13cm}Bucharest, Romania}%
        \kern 0.25 cm%
        \AND%
        \kern 0.25 cm%
        \mbox{\hrefWithoutArrow{mailto:cristiancristea00@gmail.com}{{\footnotesize\faEnvelope[regular]}\hspace*{0.13cm}cristiancristea00@gmail.com}}%
        \kern 0.25 cm%
        \AND%
        \kern 0.25 cm%
        \mbox{\hrefWithoutArrow{tel:+40-770-779-947}{{\footnotesize\faPhone*}\hspace*{0.13cm}+40 770 779 947}}%
        \kern 0.25 cm%
        \AND%
        \kern 0.25 cm%
        \mbox{\hrefWithoutArrow{https://linkedin.com/in/cristiancristea00}{{\footnotesize\faLinkedinIn}\hspace*{0.13cm}cristiancristea00}}%
        \kern 0.25 cm%
        \AND%
        \kern 0.25 cm%
        \mbox{\hrefWithoutArrow{https://github.com/cristiancristea00}{{\footnotesize\faGithub}\hspace*{0.13cm}cristiancristea00}}%
    \end{header}

    \vspace{0.5 cm - 0.3 cm}


    \section{Summary}



        
        \begin{onecolentry}
            I’m an embedded software engineer with nearly three years of hands-on experience designing, developing, and testing firmware for microcontrollers and SoCs across diverse platforms (PIC, AVR, and ARM). My work has focused on integrating robust software solutions with hardware to deliver reliable, high-performance systems, from prototype through production. Proficient in C, C++, and Python, I’ve built pipelines involving CI/CD, automated testing, and containerization (Docker, Jenkins) to ensure quality and accelerate development. I’m especially passionate about IoT and machine learning applications in embedded contexts, always eager to leverage cutting-edge technologies to solve real-world challenges. With a strong background in Agile methodologies, I thrive in collaborative settings that foster continuous improvement. I’m excited to keep learning, growing, and building innovative embedded solutions that make a tangible impact.
        \end{onecolentry}


    
    \section{Skills}



        
        \begin{onecolentry}
            \textbf{Languages:} Native proficiency in Romanian | Full professional proficiency in English | I'm learning French
        \end{onecolentry}

        \vspace{0.2 cm}

        \begin{onecolentry}
            \textbf{Programming:} C, C++ and Python | AVR, PIC, and ARM Assembly | MATLAB, Rust, TypeScript, and Java
        \end{onecolentry}

        \vspace{0.2 cm}

        \begin{onecolentry}
            \textbf{Development:} Linux | CMake and Make | Jenkins and Docker | Git and GitHub/GitLab/Bitbucket
        \end{onecolentry}

        \vspace{0.2 cm}

        \begin{onecolentry}
            \textbf{Python:} NumPy, SciPy, SymPy, Pandas, Seaborn, Matplotlib, OpenCV, scikit-image, scikit-learn, and TensorFlow
        \end{onecolentry}

        \vspace{0.2 cm}

        \begin{onecolentry}
            \textbf{Others:} LaTeX | Doxygen and Markdown | System Verilog and SPICE
        \end{onecolentry}

        \vspace{0.2 cm}

        \begin{onecolentry}
            \textbf{Interests:} Embedded Systems | Internet of Things | Machine Learning | Photography
        \end{onecolentry}


    
    \section{Experience}



        
        \begin{twocolentry}{
            Bucharest, Romania

        Aug 2023 to present

        1 year 8 months
        }
            \textbf{Microchip Technology}, \textit{Embedded Software Engineer — Full-time — Hybrid}
            \begin{highlights}
                \item Collaborated with cross-functional teams across multiple countries to design, integrate, and validate embedded system solutions, fostering efficient communication and successful project delivery.
                \item Designed and validated embedded systems for PIC/AVR microcontrollers, focusing on integration and functional testing to meet system reliability goals.
                \item Designed and implemented a pipeline system using Jenkins and Docker to perform unit testing, mocking and coverage metrics extraction within an emulation environment for embedded solution projects.
                \item Developed and updated low-level peripheral drivers for PIC and AVR microcontrollers, implementing unit tests using \href{https://www.throwtheswitch.org/unity}{Unity} and mocks using \href{https://github.com/ThrowTheSwitch/CMock}{CMock} to ensure driver reliability and functionality.
                \item Created and tested GUI-based code generation tools for new peripherals using TypeScript for development and FreeMarker for templates, leveraging an internal framework.
                \item Authored and helped with Application Notes and Technical Brief documents, including developing associated coding examples and hardware demos.
                \item Utilized DITA-based Tridion Docs to create structured documentation, ensuring consistency and compliance with Microchip’s technical writing standards.
            \end{highlights}
        \end{twocolentry}


        \vspace{0.2 cm}

        \begin{twocolentry}{
            Bucharest, Romania

        Jul 2022 to Aug 2023

        1 year 2 months
        }
            \textbf{Microchip Technology}, \textit{Intern Embedded Applications — Internship — On-site}
            \begin{highlights}
                \item Mastered embedded systems fundamentals, including microcontroller programming, interrupts, and communication protocols (UART, I2C, SPI), enabling swift integration into ongoing projects.
                \item Developed an \href{https://github.com/cristiancristea00/environmental-station}{Environmental Station} embedded project during a mandatory university apprenticeship, integrating sensor data acquisition, processing, and real-time monitoring.
                \item Gained initial experience with Agile software development methodology through the Scrum process, acquiring a practical understanding of iterative development and team collaboration.
                \item Familiarised myself with Agile-oriented tools from Atlassian, including Jira, Bitbucket, and Confluence, as well as the DevOps tool Jenkins.
                \item Contributed to the creation and updating of code examples while undergoing ramp-up training in TypeScript.
            \end{highlights}
        \end{twocolentry}


        \vspace{0.2 cm}

        \begin{twocolentry}{
            Bucharest, Romania

        Jul 2020 to Sep 2020

        3 months
        }
            \textbf{CAMPUS Research Institute}, \textit{Research Intern — Internship — On-site}
            \begin{highlights}
                \item Generated an artificial dataset of facial images with individuals wearing medical masks correctly or incorrectly, leveraging facial landmarks.
                \item Engineered and trained a neural network model using transfer learning, optimising it for real-time inference of mask compliance.
                \item Developed a Python-based application using rospy to integrate the neural network with the \href{https://pal-robotics.com/robot/tiago}{TIAGo} from PAL Robotics.
                \item Designed and implemented an algorithm to accurately extract and analyse body temperature readings from a thermographic camera feed.
            \end{highlights}
        \end{twocolentry}



    
    \section{Projects}



        
        \begin{twocolentry}{
            \href{https://github.com/cristiancristea00/sensor-board}{GitHub}
        }
            \textbf{Sensor Board}
            \begin{highlights}
                \item Designed a custom PCB featuring an STM32 microcontroller with multiple sensors interfaced via SPI, I2C, and ADC connections, leveraging KiCad for schematic capture and PCB layout.
                \item Developed the project from the ground up, following industry practices and tutorials to gain hands-on experience in PCB design, component selection, and schematic validation.
                \item Prepared the board for manufacturing and testing, ensuring design readiness for real-world deployment and validation of sensor data acquisition.
            \end{highlights}
        \end{twocolentry}


        \vspace{0.2 cm}

        \begin{twocolentry}{
            \href{https://github.com/cristiancristea00/environmental-station}{GitHub}
        }
            \textbf{Environmental Station}
            \begin{highlights}
                \item Developed an IoT-style system to monitor environmental parameters, including temperature, pressure, and humidity, using two AVR-based 8-bit microcontrollers.
                \item Designed and implemented a communication protocol between the sensor and base boards, leveraging Bluetooth modules with UART and a CRC-based data redundancy check.
                \item Integrated environmental sensors via I2C and an OLED display via SPI, focusing on efficient real-time data acquisition and visualization.
            \end{highlights}
        \end{twocolentry}


        \vspace{0.2 cm}

        \begin{twocolentry}{
            \href{https://github.com/microchip-pic-avr-examples/pic18f56q71-ultrasonic-range-detection-mplab-mcc}{GitHub}
        }
            \textbf{Ultrasonic Range Detection}
            \begin{highlights}
                \item Designed and implemented an ultrasonic range detection system using Core Independent Peripherals (CIP) on the PIC18-Q71 microcontroller family to minimize CPU resource utilization.
                \item Configured alternating PWM signals to drive a transmitter and employed a Peak Detector, Comparator, and Universal Timer (UTMR) for precise signal acquisition and time-of-flight measurement, enabling accurate distance calculations.
                \item Built and tested a functional prototype on a demo protoboard with standalone ultrasonic transducers, showcasing real-world application and hardware integration.
            \end{highlights}
        \end{twocolentry}


        \vspace{0.2 cm}

        \begin{twocolentry}{
            \href{https://github.com/cristiancristea00/tic-tac-toe}{GitHub}
        }
            \textbf{Tic-Tac-Toe}
            \begin{highlights}
                \item Developed a Tic-Tac-Toe game using the Raspberry Pi Pico, integrating an LCD screen for gameplay, a matrix keypad for user input, and a multiplexed seven-segment display for scorekeeping.
                \item Implemented three difficulty levels using the minimax algorithm for AI decision-making and a pseudo-random number generator to add variability, allowing for both single-player and two-player modes.
                \item Leveraged the RP2040’s dual-core architecture to offload user input handling, game menu control, LCD backlight management, and display brightness adjustments to the second core for optimised performance.
            \end{highlights}
        \end{twocolentry}



    
    \section{Education}



        
        \begin{threecolentry}{\textbf{M.Sc.}}{
            Bucharest, Romania

        Sep 2023 to present
        }
            \textbf{Faculty of Electronics, Telecommunications and Information Technology — National University of Science and Technology POLITEHNICA Bucharest}, \textit{Advanced Computing in Embedded Systems}
            \begin{highlights}
                \item Thesis: "Cutting-edge algorithm for detecting humans in side-by-side thermal and visible spectrum images"
                \item Achieved strong academic performance in courses such as Microcontrollers and Embedded Systems, Digital System Design, Distributed and High-Performance Computing, and Parallel Programming.
                \item Gained hands-on expertise in Reconfigurable Computing, emphasizing FPGA implementations, and deepened knowledge in Performance Analysis and Optimization and Software Development Process and Testing.
                \item Expanded technical foundation with studies in Functional Verification, Computer Vision, Machine Learning, Operating Systems, and Compilers.
            \end{highlights}
        \end{threecolentry}

        \vspace{0.2 cm}

        \begin{threecolentry}{\textbf{B.Sc.}}{
            Bucharest, Romania

        Sep 2019 to Jul 2023
        }
            \textbf{Faculty of Electronics, Telecommunications and Information Technology — National University of Science and Technology POLITEHNICA Bucharest}, \textit{Information Engineering}
            \begin{highlights}
                \item Thesis: "Degraded colour images inpainting system using deep learning techniques"
                \item Excelled in core computer science courses, including Computer Programming, Data Structures, Algorithms, and Object-Oriented Programming, with a strong focus on Microprocessor Architecture and Microcontrollers.
                \item Acquired hands-on experience in Circuit Synthesis and Testing using Verilog, complemented by a solid understanding of Digital Circuit Design principles.
                \item Built the foundation in Analog Circuit Design, including simulation and analysis using SPICE programs.
                \item Engaged in diverse coursework spanning Machine Learning, Image Processing and Analysis, Distributed and Parallel Computing, Database Systems, Digital and Statistical Signal Processing, Information Theory, Communication Systems, and Computer Networks.
                \item Developed advanced mathematical skills, including Statistics, Probability, Linear Algebra, Multivariable Calculus, and Differential Equations.
            \end{highlights}
        \end{threecolentry}


    
    \section{Certifications}



        
        \begin{samepage}
            \begin{twocolentry}{
                2021
            }
                \textbf{\href{https://www.credly.com/badges/f765af4e-e1df-4644-8a9c-4a8ab9fedc42}{CCNA: Introduction to Networks}}

                \vspace{0.1 cm}

                \mbox{Issued by Cisco}
            \end{twocolentry}
        \end{samepage}

        \vspace{0.2 cm}

        \begin{samepage}
            \begin{twocolentry}{
                2020
            }
                \textbf{\href{https://www.credly.com/badges/740a37a0-4a40-4a36-abe1-679ad53e4146}{Google IT Automation with Python Professional Certificate}}

                \vspace{0.1 cm}

                \mbox{Issued by Coursera}
            \end{twocolentry}
        \end{samepage}

        \vspace{0.2 cm}

        \begin{samepage}
            \begin{twocolentry}{
                2020
            }
                \textbf{\href{https://www.credly.com/badges/0c20b39b-18c1-4498-8e57-1aedb3f36904}{Google IT Support Professional Certificate}}

                \vspace{0.1 cm}

                \mbox{Issued by Coursera}
            \end{twocolentry}
        \end{samepage}

        \vspace{0.2 cm}

        \begin{samepage}
            \begin{twocolentry}{
                2020
            }
                \textbf{\href{https://www.coursera.org/account/accomplishments/specialization/certificate/XHAD9DTRYTMG}{TensorFlow: Data and Deployment Specialization}}

                \vspace{0.1 cm}

                \mbox{Issued by Coursera}
            \end{twocolentry}
        \end{samepage}

        \vspace{0.2 cm}

        \begin{samepage}
            \begin{twocolentry}{
                2020
            }
                \textbf{\href{https://www.coursera.org/account/accomplishments/specialization/certificate/Z79E5B7CY76C}{TensorFlow Developer Professional Certificate}}

                \vspace{0.1 cm}

                \mbox{Issued by Coursera}
            \end{twocolentry}
        \end{samepage}

        \vspace{0.2 cm}

        \begin{samepage}
            \begin{twocolentry}{
                2020
            }
                \textbf{\href{https://www.coursera.org/account/accomplishments/specialization/certificate/BRJV9TAX7QUA}{Deep Learning Specialization}}

                \vspace{0.1 cm}

                \mbox{Issued by Coursera}
            \end{twocolentry}
        \end{samepage}

        \vspace{0.2 cm}

        \begin{samepage}
            \begin{twocolentry}{
                2018
            }
                \textbf{\href{https://1drv.ms/b/s!Art6DR7ET63umGTH3dIUsOB616G7?e=Df5XSN}{Cambridge C1 Advanced}}

                \vspace{0.1 cm}

                \mbox{Issued by Cambridge English}
            \end{twocolentry}
        \end{samepage}

        \vspace{0.2 cm}

        \begin{samepage}
            \begin{twocolentry}{
                2018
            }
                \textbf{\href{https://bd.ecdl.org.ro/ecdlvcard/certification.aspx?pcode=2431704NgVYIX7Ecq9P3KujF1UR0pG}{International Computer Drivers License}}

                \vspace{0.1 cm}

                \mbox{Issued by ICDL}
            \end{twocolentry}
        \end{samepage}


    

\end{document}